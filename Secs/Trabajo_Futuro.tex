% ------------------------------------------------------------------------
% ------------------------------------------------------------------------
% ------------------------------------------------------------------------
%                            Trabajo futuro
% ------------------------------------------------------------------------
% ------------------------------------------------------------------------
% ------------------------------------------------------------------------

\chapter{TRABAJO FUTURO}
% ------------------------------------------------------------------------
% \noindent Actividades complementarias a los desarrollos presentados, incluyen el cálculo automático para conjuntos estabilizantes en plantas arbitrarias empleando el \emph{método de la signatura} desarrollado por Keel y Bhattacharyya en \myfootcite{keel2008}.\\

% Asimismo es importante explorar otras topologias de compensador y controladores PID, en sus versiones de tiempo continuo y discreto.
% ------------------------------------------------------------------------

El método propuesto puede llegar a ser validado experimentalmente mediante la implementación de un sistema óptico de adquisición de medidas cuadráticas codificadas. Adicionalmente, en este trabajo se realizó la inclusión de una capa que simula el proceso de adquisición de las medidas, por lo tanto, se abre la posibilidad de introducir una máscara de codificación $\mathbf{D}_\ell$ entrenable dentro del esquema de red neuronal, con el objetivo de determinar una codificación óptima que mejore el proceso de clasificación.
