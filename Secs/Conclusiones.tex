% ------------------------------------------------------------------------
% ------------------------------------------------------------------------
% ------------------------------------------------------------------------
%                             Conclusiones
% ------------------------------------------------------------------------
% ------------------------------------------------------------------------
% ------------------------------------------------------------------------

\chapter{CONCLUSIONES}


En este trabajo se propuso un esquema de tres etapas para la clasificación de medidas cuadráticas codificadas utilizando aprendizaje profundo. Primero, una capa simula el proceso de adquisición. Teniendo en cuenta el modelo matemático del proceso físico, esta capa contiene un modelo de propagación del campo óptico codificado siguiendo tres diferentes campos de difracción, dando cumplimiento a los objetivos específicos 1 y 3. Luego, un procedimiento de inicialización aproxima el campo óptico inicial mediante el desenvolvimiento de un algoritmo tradicional y el aprendizaje de un kernel convolucional. Por último, la red de clasificación infiere la clase correspondiente a cada medida cuadrática codificada según la estimación inicial, desarrollando con éxito el objetivo general y el objetivo específico 2 de este trabajo. Para lograr el objetivo específico 4 y validar el método propuesto, se realizaron experimentos sobre diferentes campos de difracción y redes de clasificación del estado del arte, tales como MobilNetV2, InceptionV3 y Xception. Adicionalmente, se emplearon los conjuntos de datos MNIST y Fashion-MNIST para evaluar los diferentes esquemas de aprendizaje profundo.

Los experimentos realizados exhiben un mejor rendimiento de clasificación usando el método propuesto comparado con el enfoque de clasificación tradicional que evalúa directamente las medidas adquiridas sin tener en cuenta el modelo físico. En particular, el método propuesto supera al método tradicional en hasta 0.24, 0.2, 0.25 y 0.22 en términos de exactitud, precisión, exhaustividad y métrica F1, respectivamente. Estas ganancias se reportan para la clasificación de objetos utilizando medidas simuladas del campo lejano a partir del conjunto de datos de Fashion-MNIST.