% ------------------------------------------------------------------------
% ------------------------------------------------------------------------
% ------------------------------------------------------------------------
%                             Conclusiones
% ------------------------------------------------------------------------
% ------------------------------------------------------------------------
% ------------------------------------------------------------------------

\chapter{CONCLUSIONES}
% ------------------------------------------------------------------------
% % ------------------------------------------------------------------------
% \noindent A partir de los desarrollos presentados y los resultados obtenidos en el presente trabajo de grado, es posible enunciar la siguiente conclusión general:\\
% % ------------------------------------------------------------------------
% \begin{itemize}
% \item Se analizaron las condiciones de estabilidad del conjunto de parámetros PI calculados empleando el método de diseño de controladores de \emph{Ziegler \& Nichols}. Fue posible definir una métrica para la inestabilidad del sistema controlado en el plano de parámetros $\left(k_P, k_I \right)$, a partir de una interpretación geométrica de los margenes de estabilidad del sistema. A partir de lo anterior fue posible valorar la \emph{fragilidad} del controlador PI diseñado.\\
% \end{itemize}
% % ------------------------------------------------------------------------

% \noindent De manera más puntual:\\
% % ------------------------------------------------------------------------
% \begin{itemize}
%   \item Se interpretaron las tablas de diseño de parámetros PI de \emph{Ziegler \& Nichols} en términos de conjuntos estabilizantes. Tal y como fue abordado en la \emph{Sección} \ref{estabanalpi}, se ilustró el diseño de un compensador PI para una planta y posteriormente se analizó la posición de dicho punto en el plano $\left(k_P, k_I \right)$ de controladores factibles con base en su conjunto estabilizante. A partir de ello, es claro que el método de \emph{Ziegler \& Nichols} siempre dará como resultado un controlador estable, tomando en cuenta su caracter empírico. Sin embargo, a partir de la definición de una métrica en la \emph{Sección} \ref{metrdef}, fue posible mostrar a través de una cuantificación para su \emph{fragilidad} que no necesariamente el controlador calculado es estable ante ligeras variaciones en sus valores de parámetro. De otro lado, la definición de conjunto estabilizante fue ampliamente abordada en la \emph{Sección} \ref{conjestabsect} y posteriormente aplicada al caso PI en la \emph{Sección} \ref{estabanalpi}.
%   \item Se desarrolló un algoritmo que permitió verificar las condiciones de estabilidad para controladores PI diseñados mediante el método de \emph{Ziegler \& Nichols}. Inicialmente, se realizó una discusión general de conjuntos estabilizantes en la \emph{Sección} \ref{conjestabsect}, posteriormente complementada en la \emph{Sección} \ref{conjestabpi} con medidas de inestabilidad a través de una métrica basada en la interpretación geométrica para márgenes de estabilidad en un lazo de control sometido a control PI. El método (o algoritmo) consistió fundamentalmente en calcular el conjunto estabilizante en el plano de parámetros del controlador, para posteriormente transformar las especificaciones de controladores viables a cantidades igualmente viables en el dominio del tiempo. Posteriormente un usario podría seleccionar el controlador deseado a partir de un punto en el conjunto de parámetros admisible, para el cual se provee además indicación de sus márgenes de estabilidad como medida de \emph{fragilidad}. El procedimiento anterior se desarrolló para los casos de un compensador de 3 parámetros (uno de ellos conocido) y un controlador PI.
%   \item Se implementó una interfaz para cálculo de controladores PI a partir de selección de parámetros en el dominio del tiempo, admisibles respecto al conjunto estabilizante correspondiente. El algoritmo descrito en el ítem anterior fue codificado en una interfaz en MATLAB según se describe en la \emph{Sección} \ref{interfazsect}, empleando una metodología de diseño del tipo \emph{top-down}.
% \end{itemize}
% % ------------------------------------------------------------------------