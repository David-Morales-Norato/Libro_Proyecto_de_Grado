% Introducción


\nnchapter{INTRODUCCIÓN}

Los algoritmos computacionales basados en aprendizaje profundo han sido ampliamente estudiados en la literatura, especialmente, la clasificación de objetos en imágenes ha sido una de las tareas computacionales más abordadas en ese tópico \myfootcite{li2019deep,li2018deep, wang2019development}. Los enfoques de aprendizaje profundo utilizan arquitecturas de redes neuronales, que consisten en la concatenación de múltiples capas compuestas de unidades mínimas llamadas neuronas. Cada neurona realiza una combinación lineal entre las entradas, para posteriormente usar una función no lineal en la salida. Las salidas de cada neurona en una capa funcionan como entrada de las neuronas ubicadas en la siguiente capa, creando así, una arquitectura de red neuronal profunda \myfootcite{fan2019selective}. 

En general, la clasificación de objetos se realiza sobre imágenes en escala de grises \myfootcite{bui2016using}, RGB \myfootcite{krizhevsky2017imagenet}, o más recientemente, imágenes espectrales \myfootcite{li2019deep}. Sin embargo, los enfoques de clasificación de objetos basados en aprendizaje profundo incorporan como entrada de la arquitectura de red neuronal, la información de la intensidad de la luz incidente sobre el sensor, omitiendo la información de fase, que resulta fundamental en aplicaciones como cristalografía de rayos-x \myfootcite{pinilla2018coded}, astronomía \myfootcite{fienup1987phase}, holografía \myfootcite{rivenson2018phase}, entre otras. Esta limitación se atañe a los sistemas ópticos que dependen de la conversión de fotones a electrones, puesto que no permiten una adquisición directa de la información de fase \myfootcite{shechtman2015phase}. Por lo tanto, la obtención de esta información de fase requiere la implementación de algoritmos computacionales que logren recuperar los datos perdidos.

Los algoritmos de recuperación de fase permiten reconstruir la información de un campo óptico inicial con base en la adquisición de medidas de intensidad que siguen un modelo cuadrático de propagación según diferentes campos de difracción, tales como campo cercano, medio y lejano \myfootcite{goodman2005introduction}. Dentro de estos sistemas ópticos de difracción se han incorporado máscaras de fase para la modulación del campo óptico inicial, puesto que, la literatura ha demostrado que la inclusión de este tipo de elementos ópticos durante el proceso de adquisición, genera redundancia en las medidas captadas, garantizando la recuperación de fase en hasta una constante unimodular \myfootcite{candes_CDP}. Estas medidas adquiridas a través de sistemas ópticos de difracción que incluyen máscaras de fase, se denominan medidas cuadráticas codificadas. Las imágenes recuperadas usando medidas cuadráticas codificadas se conocen como imágenes ópticas difractivas. Diversos algoritmos iterativos han sido propuestos para resolver la reconstrucción de imágenes difractivas a partir del problema de recuperación de fase. Tradicionalmente, estos algoritmos se construyen bajo formulaciones convexas, tales como el \textit{PhaseLift} \myfootcite{candes2013phaselift} y \textit{PhaseMax} \myfootcite{goldstein2018phasemax}; o formulaciones no convexas, tales como, \textit{Wirtinger Flow} \myfootcite{candes2015phase}, \textit{Truncated Wirtinger Flow} \myfootcite{chen2017solving}, \textit{Truncated Amplitude Flow} \myfootcite{wang2017solving} y \textit{Reweighted Amplitude Flow} \myfootcite{wang2018phase}.  

%\newpage
Actualmente, se han involucrado arquitecturas de redes neuronales profundas en el campo de formación de imágenes ópticas, puesto que, permite incorporar el modelo de adquisición como una capa de la misma arquitectura de red neuronal para promover una mejor reconstrucción de las medidas adquiridas. La inclusión del modelo de adquisición ha permitido el diseño de sistemas ópticos de adquisición a través de su implementación en configuraciones de red neuronal donde los pesos entrenables representan las variables optimizables del modelo de propagación \myfootcite{bacca2021transmittance}. 

Por otra parte, algunos trabajos han estudiado la clasificación de objeto usando únicamente medidas de intensidad captadas mediante sistemas ópticos definidos matemáticamente a través de operadores lineales, por ejemplo, sistemas de tomografía computarizada \myfootcite{douarre2020value}, imágenes de un solo píxel \myfootcite{bacca2020coupled}, entre otros. De manera que, los sistemas de clasificación que omiten el proceso de reconstrucción de las medidas obtenidas disminuyen el tiempo de inferencia, puesto que, se elimina la etapa de reconstrucción de la imagen, que generalmente, implica un alto costo computacional en los sistemas de clasificación. A pesar de que, se han desarrollado arquitecturas de redes neuronales para la clasificación usando medidas de intensidad bajo un modelo cuadrático \myfootcite{kim2018deep, ziletti2018insightful}, en la literatura de imágenes difractivas que resultan del problema de recuperación de fase no se han abordado modelos de redes neuronales para la detección de objetos basados en medidas cuadráticas codificadas. Así que, surge el interés de estudiar sistemas de clasificación que permitan la discriminación de objetos con base en medidas cuadráticas codificadas.

Por lo tanto, este trabajo de investigación propone el diseño de un algoritmo de clasificación de objetos en imágenes difractivas sobre medidas cuadráticas codificadas mediante el uso de aprendizaje profundo. La arquitectura de red neuronal de clasificación propuesta incluirá el modelo matemático que describe la adquisición de medidas cuadráticas para la clasificación de objetos en imágenes difractivas. Además, la evaluación de la arquitectura de red neuronal propuesta se realizará a través de bases de datos de la literatura y medidas simuladas. Asimismo, el algoritmo computacional de clasificación propuesto se comparará con técnicas del estado del arte.


Este documento se encuentra organizado de la siguiente manera: En sección 2, se describe el planteamiento del problema, incluyendo la justificación de este trabajo de investigación. La sección 3 presenta los objetivos tanto general como específicos de este trabajo. La sección 4 corresponde a la descripción de los conceptos generales y teóricos en relación con los sistemas ópticos de difracción, los algoritmos de recuperación de fase y sistemas de clasificación. La sección 5 describe la metodología planteada para alcanzar los objetivos propuestos. La sección 6 ilustra el cronograma de actividades que se realizarán durante el desarrollo del proyecto. Finalmente, la sección 7 describe el presupuesto del proyecto. 


%\pagebreak