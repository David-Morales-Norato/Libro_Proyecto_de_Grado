% ------------------------------------------------------------------------
% ------------------------------------------------------------------------
% ------------------------------------------------------------------------
%                                Anexo A
% ------------------------------------------------------------------------
% ------------------------------------------------------------------------
% ------------------------------------------------------------------------
% ------------------------------------------------------------------------
\nnchapter{ANEXOS}
% ------------------------------------------------------------------------
\anexo{Fundamentos de sólidos rígidos}\label{anexoA}
% ------------------------------------------------------------------------
Un sólido rígido, es un cuerpo formado por varias partículas puntuales que
guardan distancias constantes entre sí \myfootcite{sears2005fisica}.\\

Una operación fundamental para definir cantidades en el espacio de movimiento de
un sólido rígido es el producto vectorial (también denominado producto cruz \myfootcite{stanley1993algebra}),
el cual produce un vector perpendicular (normal) al plano formado por otros dos vectores
que se multiplican.\\

Sean dos vectores $\vec{a}$ y $\vec{b}$ definidos en $\mathbb{R}^3$. El producto vectorial entre $\vec{a}$ y $\vec{b}$ (denotado $\vec{a} \times \vec{b}$) es otro vector (digamos $\vec{c} \in \mathbb{R}^3$) cuyo cálculo puede ser efectuado a través de determinantes como sigue:
% ------------------------------------------------------------------------
\begin{equation}\label{defprodvect}
\vec{c} = \vec{a} \times \vec{b} = \begin{vmatrix}
i& j & k \\
a_i & a_j  & a_k \\
b_i & b_j  & b_k
\end{vmatrix} =
\begin{vmatrix}
 a_j & a_k \\
 b_j & b_k
\end{vmatrix}i~
-
\begin{vmatrix}
a_i & a_k \\
b_i & b_k
\end{vmatrix}j~
+\begin{vmatrix}
 a_i & a_j \\
 b_i & b_j
\end{vmatrix}k
\end{equation}
% ------------------------------------------------------------------------
De esta manera, siendo $\vec{a}=(1,-1,2)$ y $\vec{b}=(3,-4,5)$ se obtiene:
% ------------------------------------------------------------------------
\begin{equation}
\vec{a} \times \vec{b} =\begin{vmatrix}
i& j & k \\
1 &-1  &2 \\
3 &-4  &5
\end{vmatrix}=
\begin{vmatrix}
 -1&2 \\
 -4&5
\end{vmatrix}i~
-
\begin{vmatrix}
1 &2 \\
3 &5
\end{vmatrix}j~
+\begin{vmatrix}
 1&-1 \\
 3&4
\end{vmatrix}k
= 3i-j-k
\end{equation}
% ------------------------------------------------------------------------
\noindent La Fig. \ref{prodvect} ilustra esta operación gráficamente en el espacio tridimensional.
% ------------------------------------------------------------------------
\begin{figure}[h]
\centering
\caption[]{Ilustración gráfica para producto vectorial}\label{prodvect}
\includegraphics[width=0.3\textwidth]{Figs/prodvect}
% Si la figura posee una fuente distinta a los autores descomente la línea a continuación de este comentario,
% tomando en cuenta que debe realizar una cita previa fuera del caption para crear la referencia, tal y como
% lo presenta el ejemplo para la Figura \label{cuerpolibre}
% \caption*{Fuente: \arabic{footnote}.}
\end{figure}
% ------------------------------------------------------------------------

\section*{Condición de rigidez}
% ------------------------------------------------------------------------
Considere el sólido rígido presentado en la Fig. \ref{rigid}. Para cada pareja de puntos $(P_i, P_j)$ pertenecientes al sólido, se cumple:
% ------------------------------------------------------------------------
\begin{equation}\label{rigidez}
\frac{d}{dt}[\left|r_i - r_j\right|] = \frac{d}{dt}[\left|r_{ij}\right|] = 0,
\end{equation}
% ------------------------------------------------------------------------
lo cual significa que la distancia entre puntos de un sólido rígido se mantiene invariante. Esto último se conoce como la \emph{condición de rígidez}.\\
% ------------------------------------------------------------------------
\begin{figure}[h]
\centering
\caption[]{Sólido rígido}\label{rigid}
\includegraphics[width=0.3\textwidth]{Figs/rigid}
% Si la figura posee una fuente distinta a los autores descomente la línea a continuación de este comentario,
% tomando en cuenta que debe realizar una cita previa fuera del caption para crear la referencia, tal y como
% lo presenta el ejemplo para la Figura \label{cuerpolibre}
% \caption*{Fuente: \arabic{footnote}.}
\end{figure}
% ------------------------------------------------------------------------

\noindent Asimismo, a partir de \eqref{rigidez} se obtiene:
% ------------------------------------------------------------------------
\begin{equation}
\frac{d}{dt}[\left|r_i - r_j\right|] = \left|\dot{r}_i - \dot{r}_j\right| = 0,
\end{equation}
% ------------------------------------------------------------------------
y por tanto, sabiendo que $\vec{\dot{r}}$ es el vector velocidad $\vec{v}$ para un punto del sólido visto desde el observador, es posible escribir:
% ------------------------------------------------------------------------
\begin{equation}
\left|v_i\right| = \left|v_j\right|,
\end{equation}
% ------------------------------------------------------------------------
con lo cual la velocidad de traslación para cualquier punto del sólido será la misma, y así, una vez definido el movimiento de un punto cualquiera del cuerpo rigido que se traslada en el espacio, es posible definir la totalidad de su movimiento.

% ------------------------------------------------------------------------
\section*{Movimiento de rotación}
% ------------------------------------------------------------------------
En la Fig. \ref{rotacion} se ilustra un punto que rota alrededor de un eje fijo, localizado en el cuerpo del sólido.\\
% ------------------------------------------------------------------------
\begin{figure}[h]
\centering
\caption[]{Rotación de un punto del sólido alrededor de un eje fijo}\label{rotacion}
\includegraphics[width=0.3\textwidth]{Figs/rotacion}
% Si la figura posee una fuente distinta a los autores descomente la línea a continuación de este comentario,
% tomando en cuenta que debe realizar una cita previa fuera del caption para crear la referencia, tal y como
% lo presenta el ejemplo para la Figura \label{cuerpolibre}
% \caption*{Fuente: \arabic{footnote}.}
\end{figure}
% ------------------------------------------------------------------------

\noindent A partir de ello, es posible definir la velocidad angular que experimenta el punto $P$ alrededor del eje de rotación, en el modo siguiente:
% ------------------------------------------------------------------------
\begin{equation}
\omega = \frac{d}{dt}\theta
\end{equation}\
% ------------------------------------------------------------------------

\noindent Tambien, puede escribirse del diagrama la velocidad tangencial $v$ del punto mediante:
% ------------------------------------------------------------------------
\begin{equation}
\vec{v} = \vec{r} \times \vec{\omega},
\end{equation}
% ------------------------------------------------------------------------
siendo $\vec{r}$ el vector que marca la distancia del punto $P$ al eje de rotación $O$.\\

\noindent Por tanto, el vector de aceleración puede ser formulado como:
% ------------------------------------------------------------------------
\begin{eqnarray}\label{aceler}
\nonumber \frac{d}{dt}\vec{v} & = & \frac{d}{dt}[\vec{r} \times \vec{\omega}]\\
\nonumber & = & \left( \frac{d}{dt}\vec{r}\times \vec{\omega}\right) + \left( \vec{r}\times \frac{d}{dt}\vec{\omega}\right)\\
\vec{a}   & = & \vec{r} \times \vec{\alpha},
\end{eqnarray}
% ------------------------------------------------------------------------
con $\vec{a}$ y $\vec{\alpha}$ representando, respectivamente, los vectores de aceleración lineal y angular. Note que se asume $\frac{d}{dt}\vec{r} = 0$ debido a que el eje de rotación es fijo.

% ------------------------------------------------------------------------
\section*{Conservación del momento angular}
% ------------------------------------------------------------------------
\noindent En un movimiento traslacional, el principio de conservación del momento lineal establece:
% ------------------------------------------------------------------------
\begin{equation}
\frac{d}{dt}\vec{p} = \frac{d}{dt}{m\vec{v}} = 0,
\end{equation}
% ------------------------------------------------------------------------
a partir de lo cual el momento $\vec{p}$ será constante en ausencia de fuerzas externas.\\

De manera similar, es posible definir el momento angular $\vec{\mathbf{L}}$ de una partícula de masa puntual que rota alrededor de un eje fijo, en el modo siguiente:
% ------------------------------------------------------------------------
\begin{equation}\label{momang}
\vec{\mathbf{L}} = \vec{r} \times \vec{p},
\end{equation}
% ------------------------------------------------------------------------
siendo $\vec{r}$ el vector de distancia a la masa desde el centro de rotación.\\

\noindent Por tanto, el principio de conservación del momento angular puede establecerse como sigue:
% ------------------------------------------------------------------------
\begin{eqnarray*}
\frac{d}{dt}\vec{\mathbf{L}} & = & \frac{d}{dt}{[\vec{r} \times \vec{p}]} \\
                             & = & \frac{d}{dt}{[\vec{r} \times m\vec{v}]}\\
                             & = & m \frac{d}{dt}{[\vec{r} \times \vec{v}]}\\
                             & = & m\left([\vec{r}\times\frac{d}{dt}\vec{v}]+[\frac{d}{dt}\vec{r}\times\vec{v}]\right)\\
                             & = & m\left([\vec{r}\times \vec{a}]+[\vec{v}\times\vec{v}]\right)\\
                             & = & \vec{r}\times m\vec{a}\\
                             & = & \vec{r}\times \vec{F}\\
                             & = & \tau,
\end{eqnarray*}
% ------------------------------------------------------------------------
siendo $\tau$ el torque neto aplicado.\\

\noindent Empleando \eqref{aceler} puede relacionarse este torque con la aceleración angular $\vec{\alpha}$, a partir de:
% ------------------------------------------------------------------------
\begin{eqnarray*}
\tau & = & \vec{r}\times m\vec{a}\\
     & = & \vec{r}\times m\left(\vec{r} \times \vec{\alpha}\right)\\
     & = & m\left(\vec{r}\times \left(\vec{r} \times \vec{\alpha}\right)\right)
\end{eqnarray*}
% ------------------------------------------------------------------------
donde, si $\vec{r}$ es perpendicular a $\vec{\alpha}$, entonces el producto vectorial se reduce al producto de las magnitudes:
% ------------------------------------------------------------------------
\begin{eqnarray}\label{newrot}
\nonumber \tau & = & m r^2 \alpha \\
               & = & I \alpha,
\end{eqnarray}
% ------------------------------------------------------------------------
siendo $I$ el momento de inercia de las partes rotativas del cuerpo rígido.\\

\noindent La expresión \eqref{newrot} es la segunda ley de Newton de rotación, y podrá ser definida siempre que sea válido un $I$ constante. Dicha situación no siempre es posible, principalmente si se asume que el eje de rotación puede variar en el tiempo. En tal caso, $\vec{r}$ en la Fig. \ref{rotacion} no es constante y por tanto no es válida la solución propuesta para $\vec{a}$ en \eqref{aceler}, resultando en la siguiente definición alternativa para $\tau$:
% ------------------------------------------------------------------------
\begin{eqnarray*}
\tau & = & \vec{r}\times m\vec{a}\\
     & = & \vec{r}\times m\left(\left( \frac{d}{dt}\vec{r}\times \vec{\omega}\right) + \left( \vec{r}\times \frac{d}{dt}\vec{\omega}\right)\right)\\
     & = & m\left(\left[\vec{r}\times\left( \frac{d}{dt}\vec{r}\times \vec{\omega}\right)\right] + \left[\vec{r}\times\left( \vec{r}\times \vec{\alpha}\right)\right]\right)\\
     & = & m\left(\left[\vec{r}\times\left( \frac{d}{dt}\vec{r}\times \vec{\omega}\right)\right]\right) + I\alpha.
\end{eqnarray*}\
% ------------------------------------------------------------------------

\noindent El término
% ------------------------------------------------------------------------
$$
m\left(\left[\vec{r}\times\left( \frac{d}{dt}\vec{r}\times \vec{\omega}\right)\right]\right),
$$\
% ------------------------------------------------------------------------

\noindent representa los efectos (torques) debidos a las variaciones del eje de rotación, que evidentemente también representan variaciones del vector de momento angular $\vec{\mathbf{L}}$. Dichos efectos se denominan \emph{fuerzas inerciales}, puesto que tienen sentido en un marco de referencia de un cuerpo en rotación. Los tipos más representativos de fuerza inercial son los efectos giroscópicos y la fuerza de Coriollis \myfootcite{sears2005fisica}.
% ------------------------------------------------------------------------ 