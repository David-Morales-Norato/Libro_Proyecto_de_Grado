% ------------------------------------------------------------------------
% ------------------------------------------------------------------------
% ------------------------------------------------------------------------
%                                Glosario
% ------------------------------------------------------------------------
% ------------------------------------------------------------------------
% ------------------------------------------------------------------------
\chapter*{GLOSARIO}

% \begin{description}
%   \item[CONTROLADOR] (o también compensador) es un dispositivo que toma una decisión con base en la comparación de la información
%   medida con respecto a condiciones deseadas de operación. A dicha decisión se le denomina acción de control.
%   \item[CONTROLAR] es asignar valores a la variable manipulada para lograr que la variable controlada siga un valor de referencia.
%   \item[PERTURBACIÓN] señal indeseada que afecta negativamente el valor de la variable controlada del sistema.
%   \item[PID] sigla que refiere la acción combinada de control proporcional, integral y derivativo.
%   \item[SISTEMA] conjunto de elementos que interactúan de manera organizada para cumplir con un fin u objetivo común.
%   \item[VARIABLE CONTROLADA] es la cantidad o condición que se mide y controla.
%   \item[VARIABLE MANIPULADA] es la cantidad que el controlador modifica para afectar los valores de salida de la planta.
% \end{description}
% ------------------------------------------------------------------------ 