% ------------------------------------------------------------------------
% ------------------------------------------------------------------------
% ------------------------------------------------------------------------
%                                Resumen
% ------------------------------------------------------------------------
% ------------------------------------------------------------------------
% ------------------------------------------------------------------------
\chapter*{RESUMEN}

\footnotesize{
\begin{description}
  \item[TÍTULO:] \MakeUppercase{\titulo}
  \astfootnote{Trabajo de grado}
  \item[AUTOR:]\MakeUppercase{\autor} \asttfootnote{\facultad. \escuela. Director: \director. Codirector: \codirector}
  \item[PALABRAS CLAVE:] Medidas cuadráticas codificadas, imágenes difractivas, recuperación de fase, aprendizaje profundo, clasificación de objetos.
  \item[DESCRIPCIÓN:]\hfill \\ El aprendizaje profundo se ha usado ampliamente en el área de visión por computador, así como en la clasificación de objetos. Sin embargo, los enfoques tradicionales se basan principalmente en imágenes de intensidad adquiridas por sistemas de propagación lineal. Dado que los detectores ópticos solo pueden medir la intensidad del campo óptico complejo subyacente, estos sistemas lineales solo registran la información de magnitud, mientras que la información de fase no se puede medir. La recuperación de fase se ha abordado a través de sistemas cuadráticos que modulan el campo óptico usando máscaras de codificación, produciendo medidas cuadráticas codificadas. La clasificación de medidas cuadráticas se ha propuesto recientemente sin el paso de reconstrucción debido a su alto costo computacional. No obstante, la clasificación de medidas cuadráticas codificadas no ha sido estudiada en el estado del arte. Este trabajo propone un esquema de clasificación de objetos sobre medidas cuadráticas codificadas usando aprendizaje profundo. El esquema propuesto consta de tres etapas principales: una capa de adquisición que simula el proceso de adquisición; un paso de inicialización que estima el campo óptico; y una arquitectura de red neuronal que realiza la tarea de clasificación sobre la estimación inicial. Los resultados de la simulación muestran que el método de clasificación propuesto supera los esquemas tradicionales al evaluar diferentes métricas de clasificación sobre los conjuntos de datos MNIST y Fashion-MNIST a partir de los campos cercano, medio y lejano
\end{description}}\normalsize
% ------------------------------------------------------------------------ 