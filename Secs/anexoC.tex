% ------------------------------------------------------------------------
% ------------------------------------------------------------------------
% ------------------------------------------------------------------------
%                                Anexo C
% ------------------------------------------------------------------------
% ------------------------------------------------------------------------
% ------------------------------------------------------------------------
% ------------------------------------------------------------------------
\newpage
\anexo{Interfaz de animación de la dinámica del sistema}\label{anexoC}
% ------------------------------------------------------------------------
En ausencia de un prototipo real para verificar el comportamiento dinámico del sistema controlado, se optó por construir una animación que permitiera recrear el movimiento del \emph{dron} de una manera cercana al comportamiento físico real. A continuación se presentan las etapas importantes para este desarrollo.

% ------------------------------------------------------------------------
\section*{Descripción general de requerimientos}
% ------------------------------------------------------------------------
Se requiere construir una interfaz de software que permita visualizar el comportamiento del dron en el espacio de movimiento, como aproximación a la operación real del sistema para diferentes condiciones de simulación (lazo abierto, lazo cerrado controlado PID y por realimentación de estados) ante la presencia de perturbaciones. La interfaz deberá permitir modificar parámetros del sistema y los parámetros de simulación, así como también entregar información de las trayectorias de variables con respecto al tiempo.

% ------------------------------------------------------------------------
\subsection*{Nivel superior de detalle}
% ------------------------------------------------------------------------
Posterior a tener una descripción, en palabras, acerca de los requerimientos del sistema (interfaz) a desarrollar, el paso siguiente es crear un diagrama general de entradas y salidas a manera de nivel superior de detalle. Dicho diagrama se presenta en la Fig. \ref{1211}.
% ------------------------------------------------------------------------
\begin{figure}[h]
\centering
\caption[]{Representación de nivel superior de detalle para desarrollo de interfaz}\label{1211}
\includegraphics[width=0.7\textwidth]{figs/primero}
% Si la figura posee una fuente distinta a los autores descomente la línea a continuación de este comentario,
% tomando en cuenta que debe realizar una cita previa fuera del caption para crear la referencia, tal y como
% lo presenta el ejemplo para la Figura \label{cuerpolibre}
% \caption*{Fuente: \arabic{footnote}.}
\end{figure}

% ------------------------------------------------------------------------
\subsection*{Partición de primer nivel}
% ------------------------------------------------------------------------
Una primera partición se logra incorporando el bloque que realiza la solución numérica de las ecuaciones del sistema, a partir de los parámetros de entrada en el modelo y los valores que configuran la simulación, según se muestra en la Fig. \ref{1212}.\\
% ------------------------------------------------------------------------
\begin{figure}
\centering
\caption[]{Representación de primer nivel de partición para subproceso de simulación}\label{1212}
\includegraphics[width=0.7\textwidth]{figs/segunda}
% Si la figura posee una fuente distinta a los autores descomente la línea a continuación de este comentario,
% tomando en cuenta que debe realizar una cita previa fuera del caption para crear la referencia, tal y como
% lo presenta el ejemplo para la Figura \label{cuerpolibre}
% \caption*{Fuente: \arabic{footnote}.}
\end{figure}

% ------------------------------------------------------------------------
Asimismo, los resultados de este simulador serán la entrada de un nuevo bloque encargado de construir una animación para emular el comportamiento del \emph{dron} en el espacio de movimiento. Una ilustración para este segundo bloque se presenta en la Fig. \ref{1213} donde se observa también que será necesario configurar algunas opciones de simulación incorporadas como señal de entrada.\\
% ------------------------------------------------------------------------
\begin{figure}
\centering
\caption[]{Representación de primer nivel de partición para subproceso de animación}\label{1213}
\includegraphics[width=0.7\textwidth]{figs/tercero}
% Si la figura posee una fuente distinta a los autores descomente la línea a continuación de este comentario,
% tomando en cuenta que debe realizar una cita previa fuera del caption para crear la referencia, tal y como
% lo presenta el ejemplo para la Figura \label{cuerpolibre}
% \caption*{Fuente: \arabic{footnote}.}
\end{figure}
% ------------------------------------------------------------------------

\noindent Finalmente, este primer nivel de partición se completa uniendo los dos subprocesos tal y como se ilustra en la Fig. \ref{1214}.
% ------------------------------------------------------------------------
\begin{figure}
\centering
\caption[]{Representación de primer nivel de partición para desarrollo de interfaz}\label{1214}
\includegraphics[width=0.7\textwidth]{figs/cuarta}
% Si la figura posee una fuente distinta a los autores descomente la línea a continuación de este comentario,
% tomando en cuenta que debe realizar una cita previa fuera del caption para crear la referencia, tal y como
% lo presenta el ejemplo para la Figura \label{cuerpolibre}
% \caption*{Fuente: \arabic{footnote}.}
\end{figure}

% ------------------------------------------------------------------------
\subsubsection*{Partición de segundo nivel}
% ------------------------------------------------------------------------
A su vez, es posible abrir el bloque correspondiente a la simulación de las ecuaciones del sistema (según se observa en la Fig. \ref{16}) para permitir incorporar la selección del escenario de control que define una configuración importante como lo es la forzante de entrada $\Delta\boldsymbol\tau(t)$ en el modelo. De manera similar, el bloque que realiza la solución en el tiempo para las ecuaciones del sistema es un integrador numérico.\\
% ------------------------------------------------------------------------
\begin{figure}
\centering
\caption[]{Representación de segundo nivel de partición para subproceso de simulación}\label{16}
\includegraphics[width=0.7\textwidth]{figs/quinto}
% Si la figura posee una fuente distinta a los autores descomente la línea a continuación de este comentario,
% tomando en cuenta que debe realizar una cita previa fuera del caption para crear la referencia, tal y como
% lo presenta el ejemplo para la Figura \label{cuerpolibre}
% \caption*{Fuente: \arabic{footnote}.}
\end{figure}

% ------------------------------------------------------------------------
\noindent Por tanto, en la Fig. \ref{1515} se muestra el diagrama de bloques resultante para este segundo y definitivo nivel de detalle.
% ------------------------------------------------------------------------
\begin{figure}
\centering
\caption[]{Diagrama de interconexión de susbsistemas que conforman la interfaz de animación de la planta}\label{1515}
\includegraphics[width=0.9\textwidth]{figs/sexto}
% Si la figura posee una fuente distinta a los autores descomente la línea a continuación de este comentario,
% tomando en cuenta que debe realizar una cita previa fuera del caption para crear la referencia, tal y como
% lo presenta el ejemplo para la Figura \label{cuerpolibre}
% \caption*{Fuente: \arabic{footnote}.}
\end{figure}

% ------------------------------------------------------------------------
\subsection*{Selección de herramienta para implementación}
% ------------------------------------------------------------------------
A partir del diagrama obtenido en la Fig. \ref{1515}, es claro que el corazón de la interfaz a ser diseñada es el integrador numérico que resuelve las ecuaciones del sistema. Como ya ilustrado en el Anexo \ref{anexoB}, este integrador numérico ha sido codificado empleando la función $ode45$ de MATLAB. Por tanto, con el objetivo de facilitar la utilización de los desarrollos numéricos a disposición, se presenta a MATLAB como la primera opción para desarrollar la herramienta de software requerida.\\

Ahora bien, el segundo elemento importante de la interfaz es la animación que permite emular el comportamiento del \emph{dron} en el espacio de movimiento. Por tanto, aunque no es restricción que ambas componentes de la interfaz (simulador y bloque de animación) sean desarrollados en el mismo lenguaje de programación, sí se considera conveniente esta opción por motivos ligados principalmente a la reducción en tiempos de desarrollo y a una mayor compatibilidad entre componentes.\\

Adicional a esto, se recuerda que MATLAB posee además de la consola de comandos y el entorno de programación gráfico SIMULINK, un entorno para el desarrollo de interfaces de usuario denominado GUIDE (Graphical User Interface Development Environment).\\

Tomando en consideración todo lo anterior, se selecciona MATLAB \emph{vR2014a} para construir la interfaz de usuario que satisface los requerimientos de diseño ilustrados en el diagrama de nivel de partición presentado en la Fig. \ref{1515}.

% ------------------------------------------------------------------------
\subsection*{Descripción de interfaz diseñada}
% ------------------------------------------------------------------------
\begin{figure}
\centering
\caption[]{Presentación final para interfaz desarrollada}\label{siete}
\includegraphics[width=0.7\textwidth]{figs/interfaz}
% Si la figura posee una fuente distinta a los autores descomente la línea a continuación de este comentario,
% tomando en cuenta que debe realizar una cita previa fuera del caption para crear la referencia, tal y como
% lo presenta el ejemplo para la Figura \label{cuerpolibre}
% \caption*{Fuente: \arabic{footnote}.}
\end{figure}

% ------------------------------------------------------------------------
\noindent Procediendo con el diseño, se realiza codificación en MATLAB para el diagrama de bloques de la Fig. \ref{1515}, asumiendo las siguientes variables de entrada:
% ------------------------------------------------------------------------
\begin{itemize}
    \item Parámetros de la planta: [$g$, $m$, $l$, $k_\tau$, $b$, $J_G$, $I_{xx}$, $I_{yy}$, $I_{zz}$, $A_z$];
    \item Selección del tipo de escenario de control: [lazo abierto, lazo cerrado, PID, regulado espacio de estados, seguimiento espacio de estados];
    \item Parámetros de simulación: [tiempo de simulación, tiempo de perturbación, amplitud de perturbación],
\end{itemize}
% ------------------------------------------------------------------------
y de salida:
% ------------------------------------------------------------------------
\begin{itemize}
 \item Comportamiento en el tiempo de variables para sistema \emph{dron}: [posición en eje $z$; ángulos de balanceo $\phi$, cabeceo $\theta$ y guiñada $\psi$; vector de velocidades de traslación $\mathbf{v}$; vector de velocidades angulares $\mathbf{\eta}$; vector de forzantes de control $\Delta\boldsymbol\tau$].
\end{itemize}\
% ------------------------------------------------------------------------

\noindent Asimismo, se requieren los comandos de control de interfaz siguientes:
% ------------------------------------------------------------------------
\begin{itemize}
 \item Reset: para reiniciar las variables del programa;
 \item Tipo de visualización: para seleccionar el gráfico de salida a visualizar;
 \item Simular: para llamar el inicio de una simulación;
 \item Salida: para terminar el programa.
\end{itemize}\
% ------------------------------------------------------------------------

\noindent Todo lo anterior fue adecuado como se presenta en la Fig. \ref{siete}, ilustrando la presentación final de la interfaz desarrollada.
% ------------------------------------------------------------------------ 