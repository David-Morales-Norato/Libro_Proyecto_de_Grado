
\chapter*{ABSTRACT}

\footnotesize{
\begin{description}
  \item[TITLE:] OBJECT CLASSIFICATION ALGORITHM IN DIFFRECTIVE IMAGING BASED ON CODED QUADRATIC MEASUREMENTS USING A DEEP LEARNING APPROACH\astfootnote{Bachelor Thesis}
  \item[AUTHOR:] \MakeUppercase{\autor} \asttfootnote{\facultad. \escuela. Advisor: \director. Co-advisor: \codirector}
  \item[KEYWORDS:] Coded quadratic measurements, phase retrieval, coding masks, deep learning, object classification.
  \item[DESCRIPTION:]\hfill \\ Deep learning has been broadly used in computer vision such as in object classification. However, traditional object classification approaches are mostly based on intensity images acquired by linear propagation systems. Since optical detectors can only measure the intensity underlying complex optical field, these linear systems capture only magnitude information, while the phase information cannot be directly measured. Then, the phase recovery has been addressed by using quadratic systems that modulate the optical field from coding masks, producing coded quadratic measurements. Quadratic measurement classification has been recently proposed without the reconstruction step because of its high computational cost. Nevertheless, coded quadratic measurement classification has not been studied in the state-of-the-art.  This work proposes an object classification scheme over coded quadratic measurements through deep learning. The proposed scheme consists of three main stages: an acquisition layer that simulates the acquisition process; an initialization step that estimates the optical field; and a neural network architecture that performs the classification task over the initial guess. Simulation results show that the proposed classification method overcomes traditional schemes by evaluating different classification metrics over MNIST and Fashion-MNIST datasets from the near, middle, and far-fields.
\end{description}}\normalsize
% ------------------------------------------------------------------------ 